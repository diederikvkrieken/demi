\chapter{Literature Study \& Theoretical framework}
\label{ch:literature}
%Overview of Manufacturing, MAS and framework.
The manufacturing industry is and will be one of the wealth generators of the world economy. A shift towards a modular production process, the 4th industrial revolution (also called Industry 4.0 transition), results in a demand for products with high quality at lower cost while being highly customized. This results in new way of controlling the production. High-performing computing, the internet, universal access and connectivity, and enterprise integration all contribute. Overall the consensus is that only the companies that fully leverage the information, its availability, the ability to exchange it seamlessly, and process it quickly, are the companies that can meet the high demand of the consumers \citep{monostori2006agent}. 

The so-called agent-based computation is a solution for many of the problems that arise from this new trend. By having autonomous agents, who can address changes adaptively and are distributed in nature, intelligent solutions are available \citep{monostori2006agent}.

In this literature review, an overview of the manufacturing processes and current agent technologies/solutions are given. Using such a decentralized agent solution is only optimal when certain process and hardware-wise requirements, are realised on the manufacturing side. We will conclude the chapter with an overview of the framework.
\cite{abedin2014agenda}


\section{Manufacturing processes}
%Overview of manufacturing. Asset management, process control incl predictive maintenance and planningxx	, Programmable logic controller, real time control
	
	A new paradigm shift in the discrete manufacturing world requires a production that is competitive but also sustainable. Most of these solutions lie in the field of Cyber-Physical systems. A Cyber-Physical entity is one that integrates its hardware with a cyber-representation as a virtual representation. By doing so, it combines two worlds: the embedded systems and the software worlds. By doing so it breaks the traditional automation pyramid, and introduces a new more decentralized way of function \citep{leitao2016smart}. This is also visualized in figure~\ref{fig:traditional-automation-pyramid1}. %By making these systems intelligent, and connecting the design principles, the current lacking of industrial can be fully leveraged.
	
\begin{figure}
\centering
\includegraphics[width=0.9\linewidth]{"img/traditional automation pyramid1"}
\caption{The breaking of the traditional automation pyramid, and future of a new more decentralized way of function. Image from \citet{monostori2016cyber}.}
\label{fig:traditional-automation-pyramid1}
\end{figure}

	The traditional automation pyramid, is very similar to the multiple layers in the manufacturing process, which has been standardised by the American National Standards Institute (ANSI) \citep{harjunkoski2009integration}. The integration of the planning and control in the manufacturing process is one that has many aspects. Below a short overview of manufacturing will be given in the ANSI structure. This goes from asset management using process control, to real time monitoring. 
	
	
	%\Todo{UItleggen wat de 4 layers zijn en zeggen dat je ze stuk voor stuk behandelt in de volgende subsections. }
	
	\begin{figure}
\centering
\includegraphics[width=1\linewidth]{img/ansi-isa-95}
\caption{The manufacturing levels as described and defined by ANSI for the ISA-95 levels~\citep{brandl2008ISA}. }
\label{fig:ansi-isa-95}
\end{figure}

	 
%	The term scheduling is also used in various contexts and can be closely interlinked with lots of related optimization problems, such as maintenance planning, energy and inventory optimization, cutting-stock problems in the paper mills, etc. Furthermore, the time horizon largely depends on whether the focus is on short-term, mid-term or long-term scheduling or planning \citep{pinedo2005planning}. However, in this review the term planning will be used interchangeable with scheduling.	


\subsection{Asset management}
% \Todo{Niet duidelijk hoe dit met ANSI te maken heeft, \\de hoogste laag van abstractie + gramilrieit w.b. tijd.}
	Asset Management is the broad overview of the administration of assets. This includes the design, construction, use, maintenance, repair, disposal and recycling of assets. For most corporations and enterprises, the focus lies on the operational aspects of the assets, due to the fact that asset failures result in production or service delays. Therefore, insufficient asset management on one side results in loss of the asset itself, and on the other side loss due to productions delays and loss of service \citep{trappey2013multi}.  A lot is currently being researched, for example by \citet{leitao2009agent}, on asset management, and especially the condition monitoring and prediction are in focus. This is due to the shift from reactive repair work to real-time condition monitoring, prediction, diagnostics and pre-scheduled maintenance. Also, traditional asset management approaches are poorly suited for current equipment failure solutions. %\Todo{Onderhoud + inspecction. Gebruik woord ``Asset Integrity''}
	
	Traditional manufacturing control systems are unable to be sufficiently responsive, flexible, robust and reconfigurable due to the fact that they are built upon a centralised and hierarchical control structures. These are optimal for perfect optimization, but weakly responsive to change. Another consequence of this structure is that a single failure can shut down an entire system \citep{leitao2009agent}. This requires a change to decentralized asset management, demanding for new process control methods.
	
	%If the plant's availability to produce is deteriorating, then the current production schedule is easily jeopardized and rescheduling (or even replanning) may become necessary. Plant Asset Management deals with questions around plant asset health and therefore is concerned with the ability to produce as desired. Plant Asset Management (PAM) scheduling should also include plant health information in order to take care of required maintenance options, or the other way around, the maintenance schedule should be optimally embedded into the production schedule. \cite{harjunkoski2009integration}
	
	Generally, researchers use agent-based technology to represent real world situations through the use of a computational simulation process, where agents can interact with each other to find a common goal. Typically, in these environments, agents have conflicting goals. In such circumstances, they will negotiate with each other in order to resolve conflicts \citep{rosa2009intelligent}. These methods will be described in section \ref{sec:negotiation}.
		
	
\subsection{Process control}

There are three different processing methods: discrete, batch and continuous. Each process can be defined in terms of one or more of these methods. A discrete process method is when the production results in separate pieces. These are for example created in Industrial Robotic Solutions. Each robot produces a separate product in the manufacturing process. It is one of the most used manufacturing production application.

Batch production is when specific quantities of the materials have to be combined in particular ways. These are typically food productions. An example is the beer production. In a specific batch, the ingredients are combined, and after a period we have our required product. The last process method is continuous production. This type of control is required if the variables are smooth and uninterrupted in time. The process of the creation of de-mineralized water is a continuous process. The water continuously flows through the system and finalizes in the required product with no interruptions. 

An example from \citet{engell2012optimal}, which is displayed in figure~\ref{fig:processstructure} shows the typical process control method. This is in line with the ANSI standardisation described in the introduction. %\Todo{where is ANSI described}. \Todo{Hoe correspendeert dit met ANSI?}

\begin{figure}
\centering
\includegraphics[width=0.7\linewidth]{img/process_structure}
\caption{Typical process structure from \citet{engell2012optimal}}
\label{fig:processstructure}
\end{figure}

\subsubsection{Planning}
Important when controlling a process is to optimize the planning. The forms of decision making used in optimization of planning play an important role in the performance of a production plant. By using different mathematical and heuristic methods, the limited resources can be correctly allocated. This optimization is essential such that the objectives and goals of a company are satisfied (or even better). By minimizing, for example, the time to complete the production, while satisfying the goals, efficiency is increased, which often results in cost reduction \citep{pinedo2005planning}. 

One of the largest difficulties when planning, is that of ensuring that the assets are always operational, or have (as short as possible) planned downtime. This is achieved with predictive maintenance.
%Planning and scheduling are forms of decision-making that are used on a regular basis in many manufacturing and service industries. Decision-making processes play an important role in procurement and production,in transportation and distribution,and in information processing and communication. The planning and scheduling functions in a company rely on mathematical techniques and heuristic methods that allocate limited resources to the activities to be done. This allocation of resources has to be done in such a way that the company optimizes its objectives and achieves its goals. Resources may be machines in a workshop,runways at an airport,crews at a construction site,or processing units in a computing environment. Activities may be operations in a workshop,take-offs and landings at an airport,stages in a construction project,or computer programs that have to be executed. Each activity may have a priority level,an earliest possible starting time and/or a due date. Objectives can take many different forms,such as minimizing the time to complete all activities,minimizing the number of activities that are completed after the committed due dates,and so on.
%
%Generic manufacturing environment and the role of its planning and scheduling function. Orders that are released in a manufacturing setting have to be translated into jobs with associated due dates. These jobs often have to be processed on the machines in a workcenter in a given order or sequence. The processing of jobs may sometimes be delayed if certain machines are busy. Preemptions may occur when high priority jobs are released which have to be processed at once. Unexpected events on the shopfloor,such as machine breakdowns or longer-than-expected processing times,also have to be taken into account,since they may have a major impact on the schedules. Developing,in such an environment,a detailed schedule of the tasks to be performed helps maintain efficiency and control of operations. 
%The shopfloor is not the only part of the organization that impacts the scheduling process. The scheduling process also interacts with the production planning process,which handles medium- to long-term planning for the entire organization. This process intends to optimize the firm’s overall product mix and long-term resource allocation based on inventory levels,demand forecasts, and resource requirements. Decisions made at this higher planning level may impact the more detailed scheduling process directly. Figure 1.1 depicts a diagram of the information flow in a manufacturing system.

\subsection{Predictive maintenance systems}
To prevent malfunctions, maintenance is necessary. However, this maintenance results in downtime, and is preferably left out, to keep operations running. This however results in the breakdown or wear-out of these systems. By using maintaining assets before they break by so called ``preventive maintenance'' this damage can be controlled. %Predictive maintenance is the next level of solutions.

The old fashioned model is corrective maintenance. Since maintenance results in the shut-down of production plans, most companies postpone the maintenance to the last moment possible. By ensuring to take as many hours as possible from the machine, the most is taken out of their investment. However, since the breakdown can happen any moment, they need a high inventory of spare parts and materials.  And usually the repair is more expensive than maintenance. %This is typically seen as putting the cart before the horse.

Preventive maintenance is the alternative to corrective maintenance. Using predetermined fixed interval planned maintenance, the asset are maintained. However, this results in the not knowing whether maintenance is planned too early, or worse, too late. How can one be assured that the maintenance timing is optimal, due to the many factors of influence on the asset (wrong usage, or external surrounding like sun, dust and rain)? Often either maintenance is done to soon, resulting in extra cost, or to late which results in the breakdown of the asset.

Condition-based maintenance is a step in the right direction. By ensuring preventive maintenance on the right moment, the machines do not breakdown and there is no overkill on maintenance. On specific intervals, the machines are measured regarding their current status and using, for example, vibration measurements or oil samples, their current condition can be assessed. Parts that have a high probability of failure can be replaced in their next maintenance or production stop. However, this is not the optimal solution: measurements are sporadically done (not continuously) and there remains the chance of failure before the maintenance stop has occurred.

Using predictive maintenance it is possible to continuously, in real-time, monitor an installation. This can be done over a distance. Currently there are assets filled with sensors which produce data. This data is shared with people, other machines and servers. This allows for predication of failures, and real-time maintenance.

Currently a lot of research is conducted on this new form of maintenance~\citep{muller2008concept}. This central analysis is done by recognizing patterns in the data which allows for prediction of possible faults. This branch of maintenance is also known as e-maintenance \citep{yu2003multi}, condition-based maintenance or intelligent maintenance~\citep{vermaak2007multi}.

%The ability of a considered system to assess its own state in order to ask for maintenance activities is the key element to enable autonomous planning and control within the general socioeconomic maintenance system. Nevertheless, the processing of maintenance tasks is limited through specific resources, accessibility, available spare parts, tasks with higher priority, etc. These actors of the maintenance system also follow internal rules and decision logics, i.e. the mitigation of costs or risks. In conclusion, entities of the socioeconomic maintenance systems require the ability to negotiate possible constellations according to their local objectives. In complex and dynamic environments this promises to gain efficiency also on the macro level because in cooperating systems the entities pursue a globally optimized behavior and want to achieve common goals. Focusing on a multi-agent-approach in order to enable this kind of negotiation and cooperating system, the different entities of the overall system have to be modeled as intelligent agents, which are able to act without the intervention of humans or other systems \cite{weiss2000multiagent}. Each agent is self-content and acts autonomously in order to enable autonomous decision-making. Within the proposed multi-agent system the planning and control is shifted from a central system with hierarchical structures to decentralized autonomously acting agents. The general problem is split into smaller problems that agents solve locally. %\cite{Towards autonomous control in maintenance and spare part logistics- challenges and opportunities for preacting maintenance concepts}

%\cite{parunak1999industrial}


%The basic goal of all maintenance policies is to prevent or forestall future random failures of the system by removing equipments from service at an appropriate time, providing unexpected zero-downtime. Currently, many developments offer various condition monitoring techniques that directly or indirectly affect the existing maintenance policies. Data acquisition systems, signal processing techniques, experts systems, e-maintenance, and web applications make the condition monitoring much more attractive and accurate. In order to integrate all the mentioned issues within the condition monitoring techniques, agent-based technology offers an interaction level that includes coordination abilities. 



%




\subsection{Real-time Monitoring}
To ensure that processes are running accordingly and continuous planning is applied, real-time monitoring is required. Essential in implementing a real-time plan or schedule is that it has to be generated in seconds on the available computer. This may be the case if rescheduling is required many times a day because of schedule deviations. This can be done in two ways. The first way is to review the overall processes and functions performed on the data in real time, or as it happens, through graphical charts and bars on a central interface/dashboard. The second method is by implementing a programmable logic controller. By automating the industrial electromechanical processes, in a predictable and repeating sequence by use of a logic ladder, a real-time controller is achievable. 

\section*{Manufacturing with agents}
When dealing with multiple processes, production and manufacturing wise, and have to keep real-time track of the assets with sensors, the most common solution lies in agent solutions~\cite{leitao2013past, monostori2016cyber}. This is often easier said than done. In the following section, an introduction in agent-solutions will be given with a focus on negotiation and manufacturing.
%
%
%Real time monitoring for continues process is a necissaty. As explaie
%Real-time data monitoring (RTDM) is a process through which an administrator can review, evaluate and modify the addition, deletion, modification and use of data on software, a database or a system. It enables data administrators to review the overall processes and functions performed on the data in real time, or as it happens, through graphical charts and bars on a central interface/dashboard.
%Techopedia explains Real-Time Data Monitoring (RTDM)
%
% fRTDM primarily helps in monitoring and managing the consumption and use of data across a complex IT system or on a standalone software/database. Typically, an RTDM software/system provides data administrators with visual insights into the data, which is fetched from various sources including Web server logs, network logs, database logs and application usage statistics. It can also provide instant notifications/alerts into specific data-driven, administrator-specified events, such as when a data value goes out of range.
%
%%\subsection{Programmable Logic Controller}
%Used for real time control is a Programmable Logic Controller. A PLC is a digital computer used for automation of typically industrial electromechanical processes, such as control of machinery used for production. PLCs solve the logic in a predictable and repeating sequence, and ladder logic allows the programmer (the person writing the logic) to see any issues with the timing of the logic sequence more easily 
\newpage
\section{Agent Solutions}
%Overview, OOP, MAS, Holonic, Negotiation, Planning/Resource allocation, Negotiation in Manufacturing -> framework. 
The new requirements in production ask for new manufacturing planning. This requires a new planning method, which is best implemented using distributed, decentralized structures \citep{parunak1999industrial}. The basis of a distributed method lies in Object-oriented programming (OOP) and Multi-Agent structures. Using these structures and negotiations planning can be optimized.

\subsection{Object-oriented programming}

Object-oriented programming (OOP) is a programming method based on the concept of ``objects'', which may contain data and code. For example an object can be a variable, a data structure, or a function, or a combination of these. The code that an object contains can be seen as the behaviour of the object, and as such it is easily interchangeable with an agent, since a method (or message) in OOP is a procedure associated with an object. An object is made up of data and behaviour, which form the interface that an object presents to the outside world \citep{shoham1993agent}. 

Agent-oriented programming is a method often used to implement a multi-agent system. In such a system anthropomorphic ideas, like beliefs, desires are used to model the objects, and thus called agents \citep{shoham1993agent}. Agents will be discussed later in this overview.% Import to note is that based on agent-based programming, see \citet{mahar2012agent} for a thorough overview, 

% 

%they are modeled after an anthropomorphic architecture, with beliefs, desires, etc. \citep{shoham1993agent} 

\subsection{Multi-Agent Systems}

Some terms used in the literature for data collection apparatus that aggregate the data are ``Smart Objects'', ``Intelligent Gateways'', ``Collaborative Network'', ``Wireless Sensor Network'' and ``Industrial Agents''. Most of these can be viewed as Multi-Agent Systems (MAS) where the sensors communicate with one another as decentralized intelligent agents for independent action performance depending on the context, circumstances or environments (sensor input) of the situation. From such MAS, Ambient Intelligence  is conceivable: real-time decentralized decision making based on real-time data acquisition, analytics and negotiations. An example structure can be seen in Figure~\ref{fig:MAS_example}.

To define MAS, an agent needs to be defined. An agent is a system that is capable of independent action on behalf of its user or owner. As  \citet{wooldridge2009introduction} formulates it, ``An \textit{agent} is a computer system that is \textit{situated} in some \textit{environment}, and that is capable of \textit{autonomous action} in this environment in order to meet its delegated objectives.'' This independent action execution is already a form of intelligence\citep{wooldridge2009introduction}. In the MAS the developer would most probably implement such intelligence by giving each agent ``Beliefs, Desires, and Intentions'' \citep{rao1995bdi}.  

Multi-agent systems (MAS) have been identified as one of the most suitable technologies to contribute to the deployment of decentralized optimization that exhibit flexibility, robustness and autonomy\citep{vinyals2010survey}. Currently there are a lot of relevant contributions regarding agent technologies to this emerging application domain. However, many challenges remain for the establishment of MAS as the key enabling technology \citep{vinyals2010survey}. A few problems, like a lack of focus on multiple owners, decision making with only available local knowledge research and lack collective sensing strategies, are still subjects that require extensive research. They see these as the possibly most active MAS research topics. Many of these problems can be solved with negotiation, which will be covered later.

\begin{figure}[h]
	\centering
	\includegraphics[width=0.7\linewidth]{./img/MAS_example}
	\caption{Typical structure of a Multi-Agent System \citep{wooldridge2009introduction}.}
	\label{fig:MAS_example}
\end{figure}

\subsection{Holonic Systems}
Multi-agent systems are composed of autonomous software entities~\footnote{The holonic structure used in our design will be explained in the chapter~\ref{ch:design}, including a visualization}. They are able to simulate a system or to solve problems. In manufacturing the requirement linked to the real-time processes resulted in a new entity and control structure: Holonic systems \citep{giret2005multi}. A holon, just like an agent, is an intelligent entity able to interact with the environment and to take decisions to solve a specific problem. Holon has the  property of playing the role of a whole and a part at the same time. The first successfully implemented holonic structure was created by \citep{van1998reference}. PROSA consisted of three types of basic holons: order holons, product holons, and resource holons. They were structured using the object-oriented concepts of aggregation and specialisation. By decoupled the system structure from the control algorithm, logistical aspects could be decoupled from technical ones. 
\begin{figure}[h]
\centering
\includegraphics[width=0.7\linewidth]{img/holonic_manufacturing}
\caption{An example of an Holonic Manufacturing System (from \citet{giret2005multi}, based on \citet{van1998reference})}
\label{fig:holonicmanufacturing}
\end{figure}

\begin{figure}[h]
\centering
\includegraphics[width=0.7\linewidth]{img/holonic_agent_based_system}
\caption{An Holonic agent based architecture \citet{giret2005multi}}
\label{fig:holonicagentbasedsystem}
\end{figure}

The concept of holon is based on the idea that complex systems will evolve from simple systems much more rapidly if there are stable intermediate forms than if there are not; the resulting complex systems in the former case will be hierarchic. Secondly, although it is easy to identify sub-wholes or parts, ‘wholes’ and ‘ parts’ in an absolute sense do not exist anywhere \citep{van1998reference}.

\subsection{Task/resource allocation}
An example of resource allocation is when a set of agents shares a joint resource. Such a resource can be anything from indefinitely renewed continuous or discrete theoretical resource. By limiting the use of the resource to one agent at the time, negotiation is necessary to ensure that all the agents can use the resource. Usually and often crucial is the preference of the agents. Since the agents have different preferences regarding the resource, it is possible and feasible to divide the resource and create a schedule describing who has access to the resource and when \citep{fatima2014principles}. 

The same principle applies to task allocation, where the agents want to achieve a common goal. To achieve this goal quickly the agents most divide different task, which overlap, and reach an agreement on the optimal planning. 


% This paper has been concerned with how automated agents can be designed to interact effectively in both resource allocation and task distribution environments. A strategic model of negotiation has been proposed as a way of reaching mutual benefit while avoiding costly and time consuming interactions which might increase the overhead of coordination. That is, we have provided a model in which agents can avoid spending too much time negotiating an agreement and therefore are better able to stick to a timetable for satisfying their goals. In the process of developing and specifying the strategic model of negotiation, we have examined single as well as multi-agent environments, situations characterized by complete as well as incomplete information, and the differing impact of time on the payoffs of the participants. While some combinations of these factors can result in minor delays, the model nevertheless reveals an important capacity for reaching agreement in early periods of the negotiation. Throughout the paper, we have referred to two examples of application of the strategic model to problems in distributed artificial intelligence (DAI) . The resource allocation problem has been examined through the development of a scenario in which agents must share a resource in order to achieve their separate goals. The task distribution problem has been examined via a scenario in which savings can result from the sharing of tasks, and both parties benefit from cooperation. In both cases, we have met the criteria for the evaluation of a negotiation protocol which we proposed at the outset of the paper:  symmetrical distribution (no central unit or agent), efficiency (conflict avoided and no deadlocks in outcome), simplicity (process simple and efficient), stability (distinguishable equilibrium point), and satisfiability or accessibility (access to the resource or task completed). We have ended with a brief discussion of the general structure of an automated negotiator, to be based on the theoretical results of this paper. The functioning of the automated negotiator will depend upon whether it will be operating in a static or dynamic environment. The implementation of a prototype automated negotiator will provide an environment in which experimental work on the strategic model under varying initial assumptions can be undertaken, as well as one in which human negotiators can be trained. We believe that our model can be useful in other situations beside the ones we analyzed in the paper. For example, situations where there are several resources in the environment, or task distribution situations where the agents have incomplete information. We leave this for future work. \cite{kraus1995multiagent} 

%Sharing a common resource requires a coordination mechanism that will manage the usage of the resource. A coordination mechanism can be a static division of frequencies or time slots.On the other hand, it can be an on-line negotiation mechanism that dynamically resolves local conflicts over the usage of the common resource.Those are the two extreme poles of the coordination mechanism spactrum.On this spaectrum there are also coordination mechanisms that generate agreements on long term (an hour, a day, ...) global schedules. d

%As in the case of communications satellite, common resources are shared by different companies with possibly different and even conflicting goals.Therefore, to ensure efficiency, the mechanism should also be be stable and symmetric.This article formally defines these attributes and presents symmetric, stable and simple on-line coordination mechanisms that resolve local conflicts without delay and result in an efficient joint usage of the resource.

%The link between intelligent agent and game theory is established. The intelligent agents apply game theory to develop strategies that are essential in implementations of the users' tasks. Various stages of analyzing, reasoning, understanding, evaluating, predicting and executing an action must be performed by applying an optimal strategy for optimum results.

%Under some conditions, there will be convergence towards a state of equilibrium which is Pareto optimal, i.e., there exists no solution that makes some agent better off yet does not make some other worse off
%\subsubsection{Automated Negotiation}
% fipa FIPA Agent Communication specifications deal with Agent Communication Language (ACL) messages, message exchange interaction protocols, speech act theory-based  communicative acts and content language representations.
\label{sec:resourceallocation}
%Such an agents has b firstly beliefs that agent have information about their environment with may be incomplete or in correct; secondly goals, that agent will try to achieve; thirdly actions that agent perform and the effects of those actions; finally ongoing interaction that how agent interact with each other's and their environment over time. For this reason the state of an agent is called its mental state. The mental state of agents is described formally in an extension of standard epistemic logics: besides temporal the knowledge and belief operators
\subsection{Scheduling and planning}
Since most Process Planning and Scheduling (PPS) problems are NP-hard problems, many MAS have also been deployed to ``solve'' such problems in reasonable time. NP-hard (nondeterministic polynomial) problems are those problems which are at least as hard as the hardest problems in NP. This means that it is possible to reduce the problems in NP to the original problem in polynomial time. Using the decentralized global optimization approach a (sub-optimal) solution can be found. This solution would be found faster than when using an (mixed) integer program~\citep{feng2014multi}. It does however depend on the practical application of the system to see whether it is an NP-hard problem. Furthermore, this does not guarantee an optimal solution, rather that a reasonable solution will be found in reasonable time.

Real-world scheduling problems are usually complex and involve many approaches to find sub-optimal rather than optimal solutions using reasonable computing resources. The Bus Maintenance Scheduling Problem \citep{zhou2004bus}, which is distributed and dynamic in nature, has received less attention compared to scheduling problems in manufacturing. In the Bus Maintenance Scheduling Problem \citep{zhou2004bus} a MAS is proposed to heuristically solve the bus maintenance scheduling problem investigated here. It is shown that with equal optimality and less computing time without constraint violation it is comparable to the work of a mathematical programming approach.

It is also shown in \citet{bruccoleri2005production} that the agent based approach out performs the centralized mixed integer programming solution for the planning of a production.

Another example is the agile development with a MAS \citep{rabelo1999multi}. Agile development is based on the idea that requirements and solutions evolve through collaboration between self-organizing, cross-functional teams. Agile development promotes adaptive planning. By using a MAS for Agile planning, it has been shown that \textit{``the scheduling agility can be extremely improved once it is based on the following key points:
	\begin{itemize}%[i]
		\item
		distributed and autonomous systems instead of the centralized and non-autonomous solutions;
		\item 
		negotiation-based decision making instead of the totally pre-planned processes;
		\item
		application of different problem-solvers in the same environment instead of only one fixed problem solver;
		\item
		concurrent execution instead of the sequential processing''~\citep{rabelo1999multi}.
	\end{itemize}
}

Each agent is part of a heterogeneous system and processes its own information and has its own particular capabilities that it exchanges to the system. In this matter it contributes to finding a solution to the global problem which works very well in complex environments. Optimization of scheduling in such complex environments is highly constrained, with which advanced analytics also has great difficulty. Using the dynamic, flexible and intelligent relaxation of the constraints within the distributed knowledge of the agents, autonomous intelligent decision making as a Multi-Agent System is achieved \citep{rabelo1999multi}. 

% A major difficulty with classical scheduling systems is handling of conflicts. Several problems can arise during the schedule generation, after its generation, and during its execution, such as the temporal, capacity, or technologic conflicts. These problems may come from the planning, scheduling or execution supervision activities. There are several methods that can be applied for the conflict resolution in a multi-agent system. HOLOS uses the Contract-Net Protocol coordination mechanism to support the task assignments to agents, and the Negotiation [7], [17] method to overcome conflicts taking place during one of the three mentioned scheduling phases \cite{rabelo1999multi}

%\section{Agent Technologies}
\
\section{Negotiation}
\label{sec:negotiation}
Often discussed above is the negotiation of the agents in a multi-agent system. This branch of research, also called automated negotiation, is studied by Artificial Intelligence and Economics \citep{jennings2001automated}. Concepts from fields such as decision theory and game theory can provide standards to be used in the design of appropriate negotiation and interaction environments \citep{jennings2001automated}. It is used to reach an agreement that meets the constraints of two or more parties in the presence conflicting interests. And thus is a basic means of getting what you want from others \citep{fisher1987getting}. It is back and forth communications designed to reach an agreement when you and the other side have some interests that are shared, and others that are opposed. Agents reason rationally and strategically. An agent's objective is to maximize the expected value of its own payoff. 

The four components of a negotiation model are~\citep{fatima2004agenda}:
\begin{enumerate}
	\item The negotiation protocol.
	\item The negotiation strategies.
	\item The information state of agents.
	\item The negotiation equilibrium.
\end{enumerate}

Since negotiating situations occur when there is a conflict of interest, the first step will be to detect such a conflict. Agents will use communication channels and try to eliminate the conflicts. Conflicts may be about limited available resources, or may be a conflict between the beliefs of some agents. In the first case, optimization is the result, whereas, in the second case, one of the agents will have to change its beliefs \citep{shen2003multi}. Often it is seen as maximizing the quality of the result. Two solutions are possible, one, the agents can try to achieve Pareto optimality, meaning that the outcome maximizes the product of the agents' utilities, or they try to reach a Nash equilibrium, meaning an stable state in the system, both which will be discussed in the evaluation method.

Negotiation is done by exchanging messages among agents. Since the process involves several messages, a discussion will take place in which each agent's belief and goals will be an important factor. These depend on the global situation. Clearly, to be able to negotiate, agents must be able to reason. Thus, negotiation is restricted to cognitive agents. Automated negotiation is essentially a distributed search in the space of potential agreements between the different negotiators represented by autonomous agents, which involves the exchange of relevant information and aims to find an agreement that is acceptable to all participants.

Negotiation domains, can be divided into task orientated domains (TODs), state orientated domains (SODs) and worth orientated domains (WODs). TODs are the simplest and an agent's activity is defined in terms of the set of tasks it has to achieve. It is assumed that all resources are available, the benefit of negotiation is the redistribution of tasks amongst a group of agents which results in a more efficient task order. A typical example is mail delivery where an agent may carry another agent's mail at little extra cost. It is sertain that the states come closer to a Pareto optimal solution as all agents can proceed with their original task list and be no worse off. SODs deal with problem where agents wish to change their environment from an initial state to some goal state. The classic AI Blocks World problem is a good example. There is the possibility of conflict and dead end, since the agents may have different goals, and it is not feasible to try and satisfy these goals all for all agents. In this situation, agents must be able to make concessions. WODs are domains where agents attach a worth to each potential state. This allows much more flexible goals to be set and allows concessions to be made on these goals. An example would be agents in an marketplace where the goal for a seller may be to obtain the highest price for x within time y. There is again the possibility of conflict and deadlock, but now within a more complicated bargaining environment \citep{anumba2003negotiation, fatima2014negotiation} .

\subsection{Negotiation Protocol}
Negotiation Protocol is the set of rules that govern the interaction and defines who are the actors of the negotiation, the states that characterize a trade (for example, when a negotiation has begins or ends), the events that determine the change of actors’ status, and messages that can be sent by the actors in a particular state. This however is no easy task, since there is no one size fits all solution. Some attempts have been made, by \citet{marsa2014problems} for example, and a collection of design rules which allow, given a particular negotiation problem, to choose the most appropriate protocol to address it. However, these problems are only derivable when (1) the negotiation domain, including the issues and possible issue values, (2) a scenario utility histogram, which defines the distribution of contracts in utility space, and (3) several structural parameters that specify the topography (e.g. ruggedness) of each agent's utility function are known. In the design of the system, this will be discussed.

The most important protocol is that of the alternating-offers protocol \citep{rubinstein1982perfect} \todo[Rubenstein uitleggen]{Rubenstein uitleggen}. It is based on a divisible pie, discrete or continuous, and is the most widely studied among game-theorist as well as MAS researchers \citep{fatima2014principles}. Other examples is the contract net protocol and the bargaining protocol. 

A typical negotiation protocol is very similar to that of our negotiations in our everyday life and work. Thus, a negotiation typically proceeds over a series of rounds, with one or more proposals being made at each round. It also includes the rules that impose the constraints on the rules and the rule that shows when a deal has been struck \citep{fatima2014principles}. Different negotiation mechanisms need to be developed to suit the different application environments of MAS. Unlike the negotiations between human beings which involve more complex human interactions than simple technical issues, the negotiation mechanisms between agents are rule-based or case-based. Yet, the human negotiation approaches and theories, which mainly include game theory and behaviour theory, provide sound bases for the negotiations between agents. 

Another common protocol is the monotonic concession protocol. It is a proposal which has also been adapted for multi-lateral negotiation in \citep{endriss2006monotonic}.



\subsection{Negotiation Strategies}
The strategy can be defined formally as an apparatus which allows the agent to determine the content of the action that it will perform consistently with the protocol. In general, for a given set of negotiation protocol there are many strategies compatible with it, each of which can determine a different action. This means that a strategy can work well with a given protocol, but does not work with others. So, the choice of strategy depends on the protocol in use and by the trading scenario \citep{di2015multi}.

Often these strategies are private, meaning that not all the agents can see what the strategy of an agent is \citep{fatima2004agenda}. Example of a strategy is the Zeuthen strategy which results in the outcomes being equivalent to the Nash bargaining solution. 

\subsubsection{Concession Strategy}
\label{sec:concessionstrat}
\todo[Hier concession uitleggen]{Concession uitleggen!!}

\citet{endriss2006monotonic} devines 7 different concession strategies, to increase utility. 

\citet{wu2009efficient} describes 4, for resources allocaiton. 


\citet{endriss2006monotonic} A concession should always be minimal with respect to the utility loss incurred by the agent making the concession.

\citet{endriss2006monotonic} Here some concession protocols are shown, \todo[explain monotonic multilateral concession protocols]{explain the multilateral concession protocols} however, since we are dealing with private information, this is not usefull. 


\subsection{Negotiation States}
An agent’s information state describes the information it has about the negotiation game. There are two possibilities, states with complete information and those of incomplete information. The first category is basic and most common. In these games the players are assumed to know all the information about the rules of the game and the players their preferences. However, in the incomplete category, information may be lacking about a variety of factors in the problem \citep{fatima2004agenda}. %\Todo{What is the implication of this?}

%Since there is no one negotiation protocol that clearly outperforms all others in all scenarios, we need to be able to decide which protocol is most suited for each particular problem. The goal of our work is to meet this challenge by defining a “negotiation handbook”, that is, a collection of design rules which allow us, given a particular negotiation problem, to choose the most appropriate protocol to address it. \cite{marsa2014problems}
%Clearly, to be able to negotiate, agents must be able to reason. Thus, negotiation is restricted to cognitive agents
% In multi-agent systems automated negotiation is used to model the process of decision making to reach an agreement that meets the constraints of two or more parties in the presence conflicting interests. The conflict of interest refers to the fact that agents have conflicting interests, as in the case that occurs between a seller and buyer. As described in [32], automated negotiation is essentially a distributed search in the space of potential agreements between the different negotiators represented by autonomous agents, which involves the exchange of relevant information and aims to find an agreement that is acceptable to all participants. In the following the main features which characterize the automated negotiation are described: 

%Negotiation Protocol is the set of rules that govern the interaction and defines who are the actors of the negotiation, the states that characterize a trade (for example, when a negotiation has begins or ends), the events that determine the change of actors’ status, and messages that can be sent by the actors in a particular state. 
%Negotiation Object is identified by the set of parameter values which must be reached for an agreement, known as Agenda [23]. To dynamically change the agenda by adding or removing parameters may be allowed by the protocol.
%Bidding Strategies (model of decision making agent) can be defined formally as an apparatus which allows the agent to determine the content of the action that it will perform consistently with the protocol. In general, for a given set of negotiation protocol there are many strategies compatible with it, each of which can determine a different action. This means that a strategy can work well with a given protocol, but does not work with others. So, the choice of strategy depends on the protocol in use and by the trading scenario.

%Automated negotiation occurs when software agents negotiate on behalf of their human counterparts. It has been studied in artificial intelligence and electronic commerce for many years [4], [10]. Jennings et al. [5] argue that negotiation is the most fundamental mechanism to manage runtime dependencies among agents, and thus underpins cooperation and coordination. Lomuscio et al. [10] argue that automated negotiation underpins the next generation of electronic commerce systems, and develop a classification scheme for negotiation in electronic commerce. It offers a systematic basis on which different negotiation mechanisms can be compared and contrasted.

%Automated negotiation has been proposed as an ideal approach to procure cloud resources and services. Sim [16] proposes a market-driven, agent-based negotiation mechanism to procure resources and services in the cloud marketplace. In particular, the mechanism supports parallel negotiation activities, namely, multiple negotiations at the same time. It is reported that with it, agents can achieve more utility and a high success rate. Stantchev and Schröpfer [17] propose an approach for SLA mapping between business processes and IT infrastructures. It aims to formalize, negotiate, and enforce QoS requirements for cloud services. However, no negotiation approaches are specified. Yaqub et al. [29] present a generic negotiation platform for SLA@SOI (Service-Oriented Infrastructure). However, it focuses on negotiation protocols and not the negotiation strategies that we deal with in this paper. \cite{zheng2014cloud}

%anumba DOMAIN Rosenschein and Zlotkin [35] divide negotiation domains into task orientated domains (TODs), state orientated domains (SODs) and worth orientated domains (WODs), with each domain being a generalisation of the previous. TODs are the simplest and an agent's activity is defined in terms of the set of tasks it has to achieve. It is assumed that all resources are available, the benefit of negotiation being the redistribution of tasks amongst a group of agents. A typical example is mail delivery where an agent may carry another agent's mail at little extra cost. There is no possibility of deadlock as all agents can proceed with their original task list and be no worse off. SODs deal with problem where agents wish to change their environment from an initial state to some goal state. The classic AI Blocks World problem is a good example. There is the possibility of conflict and deadlock, as agents may have different goals, and satisfy all may be impossible or require more effort from each agent than if they were alone in the world. In this situation, agents must be able to make concessions. WODs are domains where agents attach a worth to each potential state. This allows much more flexible goals to be set and allows concessions to be made on these goals. An example would be agents in an electronic marketplace where the goal for a seller may be to obtain the highest price for x within time y. There is again the possibility of conflict and deadlock, but now within a more complicated bargaining environment.\cite{anumba2003negotiation} 

%anumba MECHANISM Different negotiation mechanisms need to be developed to suit the different application environments of MAS. Unlike the negotiations between human beings which involve more complex human interactions than simple technical issues, the negotiation mechanisms between agents are rule-based or case-based. Yet, the human negotiation approaches and theories, which mainly include game theory and behaviour theory, provide sound bases for the negotiations between agents. 

%anumba Besides choosing the key negotiation approaches, there are many other factors that need to be addressed based on the application scenario. Kraus [20] explains that there are five issues which need to be further determined if game theory is selected as the basic negotiation mechanism. They are: choosing a strategic bargaining model which is applicable to the problem; matching the scenarios with the game-theoretic definitions of the chosen model; identifying equilibrium strategies; developing low complexity techniques for searching for appropriate strategies; and providing utility functions.


%Negotiation is a dialogue between two or more people or parties intended to reach a beneficial outcome. This beneficial outcome can be for all of the parties involved, or just for one or some of them, in situations in which a good outcome for one/some, excludes the possibility of a desired result for the other/others. And is aimed to resolve points of difference, to gain advantage for an individual or collective, or to craft outcomes to satisfy various interests. It is often conducted by putting forward a position and making small concessions to achieve an agreement. The degree to which the negotiating parties trust each other to implement the negotiated solution is a major factor in determining whether negotiations are successful. Negotiation is not a zero-sum game; if there is no compromise, the negotiations have failed. When negotiations are at an impasse it is essential that both the parties acknowledge the difficulties, and agree to work towards a solution at a later date. Negotiation occurs in business, non-profit organizations, government branches, legal proceedings, among nations, and in personal situations such as marriage, divorce, parenting, and everyday life. The study of the subject is called negotiation theory. Professional negotiators are often specialized, such as union negotiators, leverage buyout negotiators, peace negotiators, hostage negotiators, or may work under other titles, such as diplomats, legislators or brokers.

%The study of automated negotiation play an increasingly important role in society.Since informations and tasks are shared in distributed enviornment and agents have limited functionality, it will be necessary to consider ways in which these agents can be made to interact for resourses and/or information effectively.


% Since there is no one negotiation protocol that clearly outperforms all others in all scenarios, we need to be able to decide which protocol is most suited for each particular problem. The goal of our work is to meet this challenge by defining a “negotiation handbook”, that is, a collection of design rules which allow us, given a particular negotiation problem, to choose the most appropriate protocol to address it. \cite{marsa2014problems}

\subsection{Evaluating /equilibrium solutions}
When evaluating the dilemmas of a negotiation between agents essential is the Pareto-Frontier. Visualized in figure \ref{fig:paritooptimal}, it is used to determine whether an outcome of a negotiation is efficient. This means that no improvement can be achieved for all agents. 
\begin{figure}
	\centering
	\includegraphics[width=0.7\linewidth]{img/parito_optimal}
	\caption{An example of Pareto optimal. Locations A and B are optimal, since no improvement, without loss for the agents, is possible. C and D are not optimal. From \citet{fatima2014principles}.}
	\label{fig:paritooptimal}
\end{figure}

The Nash equilibrium is the best reply to the other players strategies. This means that if both players play their Nash strategy, neither will have the incentive to change their method. Different equilibria are possible and shown in figure \ref{fig:majorcategoriesofgametheory}. 

\begin{figure}
	\centering
	\includegraphics[width=1\linewidth]{img/major_categories_of_game_theory}
	\caption{The four types of games in game theory from \citet{trappey2013multi}.}
	\label{fig:majorcategoriesofgametheory}
\end{figure}


\subsection{Principled Negotiation}
An example of a common method for negotiation is principled negotiation. This method developed by \citet{fisher1987getting} was founded on the idea that negotiators could reach better agreements by finding favourable agreements. By focussing on interests not positions and using objective criteria, an agreement is more likely to be reached. This method has successfully been deployed in a Multi-Agent System for air traffic management~\citep{wangermann1998principled}. Emphasised is the fact that it is important to agree on objective criteria for assessing options~\citep{fisher1987getting}. If an agreement can be reached using this criteria, it is more likely that it is rational. Furthermore it is useful for systems in which no agent has global knowledge of the system. 

The purpose of this type of negotiation is to help to reach agreement without jeopardizing the business relations. It was created by \citet{fisher1987getting} and they refer to this kind of agreement as a wise agreement. Wise agreement is agreement that meets the interests of both parties to the extent possible, is long lasting, and also considers the interests of the larger society. The basis of this negotiation principle is to separate the relationship issues from the problem issues, to focus on interests not on positions, while trying to be creative in developing solutions.

% \cite{wangermann1998principled} Principled NegotiationI was developed as a method that negotiators could use to reach better agreements than could be obtained using traditional confrontational tactics. The underlying idea is that in most situations there are options that will benefit all the parties in negotiation. A favorable agreement is more likely to be reached if a negotiator proposes options for mutual gain and if all the parties assess the options using objective criteria (Fig. 3). Traffic management agents would be able to a.pprove many of the proposals, as the proposing agent would try to ensure that the options provided mutual gain. 






%\subsubsection{Example}


%\todo{Breder trekken qua literatuur}%In a cooperative negotiation, agents are allowed to communicate and to receive side payments. Furthermore, in cooperative games, agents can make binding commitments. These cooperative games result in coalitions which can be used to solve some of the problems that occur from the prisoners dilemma. 

%Within negotiation there are several more methods~\cite{wooldridge2009introduction}:
%\begin{itemize}
%	\item Patient players;
%	\item Impatient players;
%	\item Negotiation decision functions.
%\end{itemize} All these methods can be used to negotiate/bargain for resource allocation. More examples are available which will also be analysed in the in-depth literature review.

%Several other communication techniques might be useful for decision making~\cite{wooldridge2009introduction}:
%\begin{itemize}
%	\item
%	Arguing;
%	\item
%	Public vs private announcement;
%	\item
%	Contract net protocol \cite{smith1980communication}.  
%\end{itemize}

%Contract net protocol is a form of cooperative distributed problem solving that will be discussed in the scheduling section (Section~\ref{sec:AIscheduling}).
%\subsubsection{Framework}
%An overview of these different methods is in Fatima et al.~\cite{fatima2014principles}. Unfortunately this book is still lacking. 
\subsection{Hierarchy and Voting}
Voting is a form of group decision making. The agents participating in the voting will take into account their own preferences as well as those of others when making decision about how to vote. This will often have a strategic flavour. By aiming to rank or order the candidates, a group decision can be made.

Another option are auctions, a popular mechanism to reach an agreement within the allocation of resources to agents. Examples include English auctions, Dutch auctions, Vickrey auctions and First-price sealed-bid auctions \citep{wooldridge2009introduction}. Interaction between a large number of low-level agents results in a complex system behaviour which is difficult to understand, to control and to predict. Structuring the agents in a hierarchy is the appropriate solution to tackle this complexity \citep{van1998reference}.

\subsection{Mapping of negotiation protocols}
An attempt at the visualization of the different negotiation techniques is strived at. Three variables are decided on. Single- vs Multi-Issue negotiation; bi- vs multi-lateral negotiation, and; perfect vs imperfect negotiation.  
\begin{figure}[h]
	\centering
	\includegraphics[width=0.7\linewidth]{img/mapping_nego}
	\caption{Current Negotiation overview}
	\label{fig:mapping_nego}
\end{figure}


\begin{figure}[h]
	\centering
	\includegraphics[width=0.9\linewidth]{img/mapping_nego2}
	\caption{Current Negotiation overview}
	\label{fig:mapping_nego2}
\end{figure}

\subsubsection{Single-Issue}

Negotiation among self-interested agents has been studied from the perspective of game theory. This is most obvious when the agents negotiate on single issues. An example might be the price of a product. When dealing with a single issue there is only one goal for both agents and there must be a conflict. If there was no conflict, no negotiation would be necessary. Typical single issue methods are patient vs impatient players, two sided matching. Argumentation based methods, which are based on the beliefs of an agents are also included in the mapping.

Essential is that all these methods are a form of the alternating offers protocol. Depending on the sort of players, the method result in completely different behaviours. These negotiation can either be complete or incomplete meaning that all information is known, or not all. 

When the game is complete, all the agents know all the information about their states and the strategies of other agents. When not all is known, the game is incomplete. The idea of negotiation is that we have an incomplete game, since if the strategies are known, most negotiation would not be necessary.

When looking at perfect vs imperfect information, it means that either the information states of the agents is perfect, meaning that the agent is perfectly informed of all the events that have previously occurred and actions (like chess), or that not all actions are known. Depending on the implementation of the system, with for example public and private announcements, the difference is made. 

When looking at single issue negotiation, depending on whether the negotiation happens between 2 (bilateral) or more (multi-lateral) agents, there are a few protocols possible. Bilateral negotiation can be either patient or impatient \citep{fatima2014negotiation} meaning that an agent has a initiative to limit the time of negotiation. Most negotiation in the manufacturing are time restrained, thus impatient agents must be implemented \citep{kraus1995multiagent}. In symmetric vs asymmetric the players are uncertain about the other players utility functions (as is the case in imperfect negotiation), but essential is that one agent might know more than the other in the asymmetric protocol.

\subsubsection{Multi-Issue}
 When negotiating multi-issues, agents attempt to combine 2 or more issue in their discussing. An example is the typical seller, buyer relationship between two agents, as for example shown in \citet{schramm2013bilateral}. Here a supply chain construction company is used to asses an method to support bilateral negotiation. Aspects like price, quality and lead-time are considered as issues, on which can be negotiated. Most used multi-issue method, for single-lateral negotiation is the package deal method. In this method, complete packages with all the issues are provided. These can be discussed either sequentially or simultaneously. 

 Agents can employ either an issue-by-issue (one-at-a-time) approach, or a packaged approach in the negotiation agenda \citep{fatima2004agenda}. In \citet{abedin2014agenda} a packaged approach for this is lack of knowledge about the opposing agents. As one issue is settled, the agent subsequently negotiates the other pending issues. This allows the agent to be cautious and opportunistic at the same time. For a multi-issue negotiation under incomplete information settings, the ideal solution is one that is Pareto optimal. A solution is said to be Pareto optimal if no agent can be better off without sacrificing the other’s utility as will be discussed in the evaluation.
 
 \begin{figure}
\centering
\includegraphics[width=0.7\linewidth]{img/multi-lateral}
\caption{An overview of the different negotiation method for multi issues bargaining. From \citep{abedin2014agenda}.}
\label{fig:multi-lateral}
\end{figure}
When choosing the preferred method of negotiation, important to realize is the solution required. As explained above, the issue-by-issue approach has a higher chance of optaining the Pareto optimal solution. Furthermore, the majority of the existing work on multi-issue negotiations focuses on the negotiation strategy, assuming the agenda and the procedure to be predetermined \citep{fatima2004agenda, lai2004literature}. Interesting to determine would be the influence of the domain and protocol since, depending on the scenario under which the negotiation is taking place a supervised agenda procedure can have a positive impact on the outcome of the negotiation when compared to a procedure without use of an agenda. \citep{abedin2014agenda}.


% 
 %The agenda based approach specifies how the issues will be settled. Agents can employ either an issue-by-issue (one-at-a-time) approach, or a packaged approach in the negotiation agenda (Fatima et al. 2004). We have adopted the former; negotiation one issue at a time. The main reason for this is lack of knowledge about the opposing agents. As one issue is settled, the agent subsequently negotiates the other pending issues. This allows the agent to be cautious and opportunistic at the same time. For a multi-issue negotiation under incomplete information settings, the ideal solution is one that is Pareto optimal. A solution is said to be Pareto optimal if no agent can be better off without sacrificing the other’s utility (Wilkes 2008). So the proposed negotiation approach should be able to generate Pareto optimal solutions for multi-issue negotiations. The majority of the existing work on multi-issue negotiations focuses on the negotiation strategy, assuming the agenda and the procedure to be predetermined (Fatima et al. 2004; Lai et al. 2004). Depending on the scenario under which the negotiation is taking place a supervised agenda procedure can have a positive impact on the outcome of the negotiation when compared to a procedure without use of an agenda. \cite{abedin2014agenda}

% Issue by issue negotiation gives a high probability to find a negotiation zone in an incomplete information setting rather than package and simultaneous approach. In package and simultaneous negotiation approach, the agents have limited confident and information about each other, the issues, offers and preferences. With issue by issue negotiation, there can be different agendas. Generating the optimal agenda for both agents helps to maximize their utilities. A simple but useful interdependency mechanism is derived to handle the correlation between the issues. \cite{abedin2014agenda} 

% First, in a multi-attribute negotiation the preference of an agent over multiple issues can be complex. A traditional way to deal with this is to characterize the preference with a utility function (a mathematical formula) and agents make decisions based on this utility function. However, it is not trivial for a human to construct such a utility function over multiple issues, especially when preference over one issue is impacted by the values of other issues; thus, preference elicitation may take a long time or sometimes be intractable. Second, in a multi-attribute negotiation the solution space is n-dimensional (n>1) rather than a single dimensional line as in a single-attribute negotiation. This makes the negotiation strategy in multi-attribute negotiations complex: because the space is ndimensional, every time an agent plans to concede, she needs to first decide the direction of concession. Apparently there are many choices on the concession direction she can take: to concede on issue 1, …, n or different combinations of the issues. Specifically, the decision on the concession direction may also depend on the opponent’s preference because conceding on the issue more important to the opponent can make the offer more acceptable. Finally, to decide how much to concede is now more complicated because the direction can impact the amount as well. So the burden of computation and reasoning for the negotiation strategy is higher in a multi-attribute negotiation than in a single-attribute negotiation. Third, as mentioned above, in multi-attribute negotiations there exist “Win-Win” situations. For rational agents, they should not “leave extra money on the table”. In other words, the ideal result for the system is to realize a Pareto-optimal (or Pareto-efficient) solution. A Pareto-optimal solution is one which can not be improved further without sacrificing someone’s utility, i.e. if there is another solution from which one of the agents can get more than from this Pareto-optimal solution, then the other agent must get less by that other solution. We say a multi-attribute negotiation model is efficient when agents will reach a Pareto-optimal agreement in the negotiation, if there exists a zone of agreement. \cite{lai2004literature}

\subsubsection{Multi-lateral}
The most common used method for multilateral negotiations are contract based methods, most popular being the contract net protocol. Contract net protocol by \citet{smith1980communication} is based on the principle that agents, each with a distinct expertise, can solve sub problems that are required to solve the global problem. This form of cooperative distributed problem solving is based on the assumption that agents in a system implicitly share a common goal, and thus that there is no potential for conflict between them.

Each agent (manager) having some work to subcontract broadcasts an offer and waits for other agents (contractors) to send bids. After some delay, the best offers are retained and contracts are allocated to one or more contractors who process their subtasks. The contract-net protocol provides for coordination in task allocation. 

The protocol is best suited to problems in which it is appropriate to define a hierarchy of tasks. Such problems lend themselves to decomposition into a set of relatively independent subtasks with little need for global information or synchronization. Individual subtasks can be assigned to separate processor agents. The main contribution of the contract net protocol is the mechanism it offers for structuring high-level interactions between nodes for cooperative task execution. Negotiation can be used at different levels of complexity. At one extreme, it is a means of achieving task distribution with distributed control and shared responsibility for tasks to maintain reliability. At the other extreme, the twoway transfer of information and mutual selection attributes of negotiation  make possible a finer degree of control in making resource allocation and focus decisions than is possible with traditional mechanisms \citep{smith1980communication}.

Since the contract net protocol has the uncertainty of matches being stable, the protocol of two-sided matching has been developed. Furthermore it is not certain that the matches are Pareto optimal. Using the two sided matching method, this uncertainty can be avoided, however, this protocol is harder to implement due to the fact that a clear allocation division is required. \citep{fatima2014principles}.
 
If the game is imperfect two sided matching does not work, and a proposal based protocol is the right fit \citep{rahwan2003argumentation}.


%\subsubsection{Reasoning about ones interest}
%Interest-based negotiation (IBN) is a form of argumentation-based negotiation in which agents exchange (1) information about their underlying goals; and (2) alternative ways to achieve these goals \cite{rahwan2009formal}. 


\subsubsection{Heuristic methods in negotiation}
Most of the negotiation in manufacturing can be seen as multi-lateral multi-issue negotiation. Three important distinctions are to be made, based on \citet{lai2004literature}. 
\begin{enumerate}
	\item
	issue by issue negotiation;
	\item
	multi-issue cooperative negotiation;
	\item
	multi-issue negotiation with heuristic methods.
	\end{enumerate}
	
	The first aspect looks at the agreement which is built through a strategy, and examines this individually and interactively, and the parties are considered as non-cooperative and they are built for environments with incomplete and asymmetric information, where an agenda containing the order in which issues are treated is needed. For the second aspect a multi-issue concession strategy is used whose parties are considered cooperative and they have complete and symmetrical information about their environments. These two aspects have been discussed in the sections above. In the last type, an agreement is reached through a hybrid negotiation strategy, which uses the first two types of theoretical framework with the focus in automated models based on autonomous agents for multi-issue negotiation and in negotiation strategies tractable. This is also where possible learning methods are available \citep{schramm2013bilateral}. 
	
	These heuristic methods are a lot more common in the implementation of negotiation, as discussed by \citet{leitao2013past, monostori2006agent}, since it does not require the through analysis of the states and protocol compared to the game theoretic methods. Also it allows for larger groups and learning in the agents.
	
\subsubsection{Learning methods in Negotiation}
When dealing with heuristic methods for negotiation, learning methods can be implemented. An overview can be seen in figure~\ref{fig:negotiationlearning}. 

\begin{figure}
\centering
\includegraphics[width=0.7\linewidth]{img/negotiation_learning}
\caption{Overview of different learning methods for heuristic negotiation methods from \citet{beheshti2014homan}}
\label{fig:negotiationlearning}
\end{figure}
Based on the research conducted on heuristic methods and \citet{jennings2001automated}, it can be concluded that the optimal research in learning in heuristic methods is not yet known. They are often used however to decide on the optimal counter bid in \citet{beheshti2014homan}. They show that efficient learning algorithms based on an statistical ranking algorithm and linear regression, all with linear time complexities. These characteristics allow our method to be used in real-world applications. 
\section{Negotiation in Manufacturing}
There are many applications of agent based solutions in the manufacturing world \citep{monostori2006agent}. In these negotiations an overwhelming aspects is realised in the creation of intelligent individual agents, and less on the overall intelligence of the system. Often ignored is the specific negotiation method in these systems. This is where the problem lays, since conflicting interest, essential in the optimal decision making are left out. An example where these conflicting interest are well implemented is in \citet{zheng2014cloud}. A cloud consumer usually prefers a high reliability, whereas a cloud provider may only guarantee a less than maximum reliability in order to reduce costs and maximize profits. If such a conflict occurs, a Service Level Agreement cannot be reached without negotiation. Automated negotiation occurs, when software agents negotiate on behalf of their human counterparts. However no learning occurs. 

%Many negotiation solutions have been ``invented'' but little has been truly applied in the manufacturing world \citep{leitao2009agent}. 

Rockwell Automation uses agents in its automation processes and is one of the industrial leaders in the implementation of agent based solutions \citep{vrba2011rockwell}. One of their future insights in the requirements of agent based solutions is to enhance the capabilities of agents for expressing and exchanging knowledge, and as a consequence, to increase the flexibility of control systems. In order to correctly do so, better insights in the negotiation is needed.

Overall, nearly all factory scheduling negotiations use some form of these market-based approaches \citep{monostori2006agent} to implement the solutions. Different version of the contract net protocol were used or other auction based methods. The problem with these methods is that no reasoning about another's interest and desires is achievable. If this is known, more efficient and better systems can be achieved. It is however shown in \citet{bruccoleri2005production} that the agent based approach using market auctions out performs the centralized mixed integer programming solution. This system uses bilateral simultaneous negotiation on the medium level of the production plant. It is however a form of auctions, where the agents simultaneously bid towards the goal. If this system already outperforms a centralized system, a non-auction based method might outperform even better.

Other examples of negotiation in a Multi-Agent System have been deployed in Smart Grids for optimal energy delivery~\citep{pipattanasomporn2009multi}, the collaborative design of light industrial buildings~\citep{anumba2003negotiation}, negotiation in an electronic market of water rights, and for example in the scheduling of Agile software development~\citep{rabelo1999multi}. 

From the above, in comparison with the knowledge obtained, there are two gaps. Firstly, little multi-issue multi-lateral strategic (game theory wise) application have been implemented. An example from the theory is \citet{wu2009efficient} where a Pareto-optimal-search method for three-agent multilateral negotiation is developed. This however has not been implemented in any real usecase, and would be very interesting to implement. The other gap in the literature is the research into the optimal learning methods for heuristic methods. In \citet{de2015automated} a wireless surveillance sensor networks is optimized using heuristic learning methods. This is limited to a bilateral negotiation protocol with a mediator, where negotiating agents (two access providers, each of them controlling a fraction of the access points in the scenario) negotiate. No multilateral application has been attempted. An attempt at generalizing multilateral heuristic learning has been made in \citet{beheshti2014homan}, but this has not been applied to a real use case as of yet.

The last option, an application of multilateral heuristic learning, is the best fit on the usecase.
%So, in future, more concrete work should be done to validate more globally the decision-making support system and to find the most appropriated architecture ensuring “a form” of hierarchy (to optimise the global performances) and of co-operation (to be locally more reactive). It can lead (a) to evolve towards benchmarking, will not thus limit “to evaluate” only the protocol of negotiation we presented but also some of its derivatives and other types of negotiations, (b) to “automatise” the utility function of each expert (automated human expert) based on techniques such Bayesian Network, Fuzzy Logic,…to reduce the complexity of each decision (and to help the expert to develop his proposal), and (c) to use formal proofs to prove the global behaviour properties of the e-maintenance architecture (dynamics of the agent interaction and of the result emergence).  \cite{yu2003multi}

%In \cite{harjunkoski2009integration} it is stated that  Even if it is today technically possible to connect every single device within a control system into an plant-wide MES system, it would make very little sense to integrate these e.g. with the scheduling. This is due to traditional view of a centralized system for planning. What this fails to see is the positive effect it would have if 
% An Adaptive Negotiation Framework for Agent Based Dynamic Manufacturing Scheduling 
%Manufacturing scheduling problems, especially in a dynamic, uncertain environment, belong ta the NP-hard class. Agent based approach has been considered as a promising solution to these types of problems. However, it brings in another challenge which is how to make those costly negotiation processes among agents effective. This paper addresses this problem by proposing an Adaptive Negotiation Framework which models the dynamic nature of agent based manufacturing scheduling at negotiation level explicitly, The main contributions of this paper include: (I) an Agent-Based Adaptive Negotiation Framework for manufacturing scheduling, (2) selection heuristics of multiple economically inspired negotiation models, (3) the integration of adaptive negotiation heuristics, economic models and the coordinated Intelligent Rational agent architecture. 
%
%In the category of multi-agent systems (Bonabeau \& Theraulaz, 2000; Wooldridge, 2002; Brussel \& Wyns \& Valckenaers \& Bongaerts, 1998), ontology based optimizers were preferred ahead of generic particle swarm optimizers, as ontology based systems attempt to assess the consequence of mutation of the existing solution, prior to mutating, whilst generic PSO's mutate and then assess the fitness of the solution in the solution landscape. For practitioners this means that ontology based systems have fewer mutations though the run-time is comparable with PSO's. Finally amongst the different categories of ontology based systems, negotiating resource-demand-networks have shown to be most efficient (Rzevski \& Skobelev,2007; Skobelev, 2011; Multi-Agent Technology web-site (2012).
%
%An overview of agents in manufacturing is given by \citet{monostori2006agent}, where he gives an thorough and complete overview. In this paper a focus lies on the negotiation in a Multi-Agent solution, and thus the focus lies here as well. 
%
%\citep{shen2003multi}
%%\section{Artificial Intelligence}
%The research will be based on an intelligent Multi-Agent System (MAS) which would consist of sensors, connected as an Internet of Things (IoT) network. For the intelligent agents it would be possible, by understanding the system, and negotiating, to come up with the optimal inspection and maintenance schedule based on real-time data acquisition, analysis, negotiations and decentralized autonomous decision making. Such intelligence is an example of a typical MAS where artificial intelligence may include methodical, functional, procedural approach, algorithmic search and/or reinforcement learning. 
%%\subsection{Internet of Things}
%%
%% %%%%%%%%%%%%%%%%%%%%%%%%%%%%%%%%%%%%%%%%%%%%%5


%cite{anumba2003negotiation}
%Although there are essential differences between agent negotiations and human beings negotiations, the negotiation theories developed based on human activities are also the basis for agent negotiations. The adoption and development of negotiation mechanism depends on the application environment, requirements, and expectations of MAS.

%Game theory has a history of being applied to human or animal behaviour, but one of the main criticisms for the application of game theory has been that animals and, in particular, humans exhibit much more complicated behaviour than these simple models predict; human beings often do not act rationally and frequently do not have consistent preferences over alternatives. Machines, on the other hand, are governed by more consistent preferences, have generally very simple goals and motivations, and are more predictable. Game theory would seem to lend itself to use by intelligent agents and provides a basic negotiation mechanism for most MAS.

%However, there are problems. Much of the basic intelligent agent game theory research has been on ‘toy’ problems such as Blocks World, where a number of blocks placed in known slots, or on each other, have to be rearranged. In such problems the domain is small and all outcomes can easily be evaluated. Much of the theory is focussed on deriving conditions under which the most optimal deal is always reached. In the real world, these conditions can rarely be met and criticism has focussed on the many unrealistic assumptions of these approaches [27]. As a result, the axiomatic approaches are not suitable for the complex agent negotiation systems because they are basically static approaches, where the outcomes of the negotiation are emphasised.

%On the other hand, the strategic approaches, as a dynamic approach, overcome some of the major problems of the axiomatic approaches. Since the strategic approaches focus on the negotiation process as well as the outcomes of negotiations, such approaches are more suitable to the cases where the negotiation process is important. A typical example is the negotiations about the wage between the management group and the union. How one party makes a concession is important for another party to make their decision.

%Although the strategic approaches could be adopted for the design of agent negotiation mechanisms, many criticisms to the approaches are still very real when they are used to handle the real world problems. A major consideration is human responses during negotiation. Being more realistic and effective, behaviour models often provide interesting aspects for the development of agent negotiation mechanism. One of the examples is the learning mechanism which allows agents to get more realistic ideas about the opponent's negotiation strategy based on the opponent's offer during the negotiation process [51]. Some simple mechanisms such as the Bayesian rule allows intelligent agents to adopt such a learning ability. Beside learning mechanism, other behaviour models such as the dual response model and the joint decision making model are also applicable to building a sophisticated negotiation mechanism for MAS.

%Besides choosing the key negotiation approaches, there are many other factors that need to be addressed based on the application scenario. Kraus [20] explains that there are five issues which need to be further determined if game theory is selected as the basic negotiation mechanism. They are: choosing a strategic bargaining model which is applicable to the problem; matching the scenarios with the game-theoretic definitions of the chosen model; identifying equilibrium strategies; developing low complexity techniques for searching for appropriate strategies; and providing utility functions.



%approximate-optimale oplossingen terwijl je de optimale oplossing niet weet lijkt een leuke uitdaging
\newpage
\section{Framework}
A framework to compare a centralized vs a decentralised solution is discussed here. Essential in the difference between these two possible solution spaces is the location of the processing power for the calculations. Centralised solutions have a single control unit where the information flows to, while decentralized solutions do not have this structure.

A popular comparison, discussed in \citet{parunak1999industrial}, is that of the original Roman army structures. Decisions where made at the top and dripped down, while the information stream went up. This method, has been deployed into most companies. Due to the fact that something can be computed on a single computer, and be optimized on this single program, an optimal can be found.

However, the increasing complexity of computer and information systems, combined with the increasing complexity of their applications, exceed the level of conventional centralized computing. This is due to the processing of huge amounts of data, or data that origin from (geographically) distinct locations. To solve such difficulties computers have to act more as agents where each agent can solve part, or decides on part of the problem. This is where agent-based architectures are an ideal fit to such an organizational structure.

To push the decision making to the lowest level, excessive layers of management can be absolute. This allows for, sometimes, easier to understand and developing of problems, especially if the problem being solved is itself distributed.

By using principles of decomposition which is a classical optimization (reformulation) method \citet{sharif2012yard} presents a comparative study of two contrasting approaches for modelling the yard crane scheduling problem: centralized and decentralized. It seeks to assess their relative performances and factors that affect their performances. They conclude that a centralized approach outperforms the decentralized approach by 16.5 \% on average, due to having complete and accurate information about future truck arrivals. However, since the decentralized under performs the centralized, the decentralized approach can dynamically adapt to real-time dynamic changes, making it better suited for real-life operations. 

To optimize these different types of resources allocation problems, there are different kinds of allocation problems, for which different solutions are feasible. The purpose here is to find what characteristics are optimal to use a centralized vs a decentralized solution.

%\begin{figure}
%\centering
%\includegraphics[width=0.7\linewidth]{img/yard_compare_ca_da}
%\caption{}
%\label{fig:yardcomparecada}
%\end{figure}



% The increasing complexity of computer and information systems goes together with an increasing complexity of their applications. These often exceed the level of conventional, centralized computing because they require, for instance, the processing of huge amounts of data, or of data that arises at geographically distinct locations. To cope with such applications, computers have to act more as \individuals" or agents, rather than just \parts."

%centralized solutions are generally more efficient: anything that can be computed in a distributed system can be moved to a single computer and optimized to be at least as efficient. However, distributed computations are sometimes easier to understand and easier to develop, especially when the problem being solved is itself distributed. Distribution can lead to computational algorithms that might not have been discovered with a centralized approach. There are also times when a centralized approach is impossible, because the systems and data belong to independent organizations that want to keep their information private and secure for competitive reasons.


\subsection{Size and modularity}
A \Todo{Framework weghalen} critical aspect of the possibility to determine whether a centralized or decentralised solutions is preferred is the search space size of the problem. The size of the problem is seen as the number of resources that have to be allocated.  If a clear structure is conceivable and a clear population is in place a centralized solution is feasible. This is due to the global overview . 

%Since a centralized solutions are highly sensitive to the size and complexity and require performing the computation in advance, they lack the dynamically needed in real-life situations. The decentralized Approach dont require computation time required in advance and can use disaggregated, handle large and complex problems

%\subsection{Modularity}
In a decentralized structure, individual models are decoupled from one another, errors in one module impact only those modules that interact with it, leaving the rest of the system unaffected. This can be seen in figure~\ref{fig:modularitydecentral-changeability}. It shows however the importance of having a clear modular problem. 

%As Figure 9.1 suggests, in a system with a single thread of control, changes to a single module can cause later modules, those it invokes, to malfunction. Decentralization decouples the individual modules from one another, so that errors in one module impact only those modules that interact with it, leaving the rest of the system unaffected. (The original version of this  
gure was created by Seiichi Yaskawa of Yaskawa Electric Corporation, Tokyo, Japan, and is used with his kind permission.)

\begin{figure}[h]
\centering
\includegraphics[width=0.7\linewidth]{img/modularity+decentral-changeability}
\caption{Comparison of a conventional control thread and an agent based, from \citet{parunak1999industrial}.}
\label{fig:modularitydecentral-changeability}
\end{figure}



\subsection{Dynamicity (Time Scale/Changeability)}
In a centralized solution, the continuous monitoring of the state of the environment and typically the lack of complex decisions, a quick reaction to changes is possible. A high dynamical is the result. 

Unfortunately, it is difficult to achieve real-time scheduling in traditional manufacturing systems because the scheduling algorithms used are executed on a single, centralized computer that becomes a computation \citep{duffie1994real}.
\subsection{Solution Quality}
Since agent-based approaches are distributed, they do not have a global view of the entire state of a system. A lot can reached through communication and negotiation, but for a truly optimal solution, an entire view is necassary. For example,  \citep{palmer2003decentralized} shows that this algorithm is not intended to find the optimal solution; it finds a good solution with less computation. 

In the centralized approach the assumption of a complete information on supply and demand is made. This requires rescheduling to adapt with changes. 

In the decentralized approach, no assumptions on the are necessary. 
%Adaptability
%Centralized Approach
%Assumes complete information on supply and demand
%Requires rescheduling to adapt with changes
%Decentralized Approach
%No assumptions on the arrival-time of trucks
%Monitor changes continuously, adapt rapidly


\subsection{Complexity}
Requirments are Simple relations. %However when changing relations are  %Changing relations a



\begin{tabular}{p{2cm}|p{3cm}|p{3cm}|p{3cm}}
	& Centralised Solution & Decentralised Solution & Building Blocks\\
	\hline \hline
	Size / Modularity & Small;  No sub-problems & Large; Ill-Structured; Easily dividable; Independent Modules & Population; Holonicicity; \# of resources: (decision variables, parameters \& constraints)\\
	\hline
	Time scale / Changeability & Days - Weeks; Not subject to a lot of change & Real-time - Hour; Changeability  & Adaptive Capability ; Degree of Re- and Pro-activeness\\
	\hline
	Solution quality & Perfect & (sub-)Optimal & Object and Solution Space\\
	\hline
	Complexity & Simple & Complex & Interaction between the set of elements; Communication
\end{tabular}
%\cite{pujolle2006autonomic}
%To achieve the autonomic-oriented architecture, we propose to select the appropriate control mechanisms among: 
%- adaptive: the agent adapts its actions according to the incoming events and to its vision of the current system state. The approach we propose is adaptive as the agent adapts the current control mechanisms and the actions undertaken when a certain event occurs. The actions the control mechanism executes may become no longer valid and must therefore be replaced by other actions. These new actions are indeed more suitable to the current observed state;
%- distribution: each agent is responsible for a local control. There is no centralization of the information collected by the different agents, and the decisions the agent performs are in no way based on global parameters. This feature is very important as this avoids having bottlenecks around a central control entity;
%- local: the agent executes actions on the elements of the node it belongs to. These actions depend on local parameters. However, the agent can use information sent by its neighbours to adapt the activated control mechanisms;
%- scalable: the proposed approach is scalable because it is based on a multi-agent system which scales well with the growing size of the controlled network. In order to adaptively control a new node, one has to integrate an agent (or a group of agents) in this node to perform the control. 
