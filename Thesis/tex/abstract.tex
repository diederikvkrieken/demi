%abstract
\begin{abstract}
	In this we study how negotiation in a Multi-Agent System (MAS) can be used in business and production context.
	
	Firstly, an overview of production \& manufacturing is given, after which the possible agent solutions are discussed. Here we determine the specific requirement for business if one where to implement a MAS in a process to optimize. Afterwards we cover the different possible negotiation techniques that can be used for the agents to communicate. Here a knowledge gap in multi-issue multi-lateral negotiation, with private utility functions is found. Using the alternation projection method to negotiate, process optimization should be possible. This is tested with an usecase, and using the reactive compared to the non-reactive concession strategy the optimal concession strategy is discussed. It is found that the reactive concession strategy is not as well performing as the non-reactive to the optimal (Nash and Pareto) solution, since it can stall while the agreement-zone is non-empty. However, if a single agents uses the reactive strategy, the system performs well.  We conclude with a possible solution for the initial problem and future research steps to be performed.
\end{abstract}