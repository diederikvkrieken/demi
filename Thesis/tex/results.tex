\chapter{Simulation comparison \& evaluations}
The simaltation run is that of the Design

\section{Evaluation}
At the moment, it seems as if the reactive concession strategy, as described in \citet{zheng2015automated} still has some difficulties. This can be clearly seen in \cref{fig:solution-mixbed-neut}. 
\begin{figure}[h]
	\centering
	\includegraphics[width=0.7\linewidth]{"img/solution mixbed neut"}
	\caption{The current search for a solution for the mixbed agent and the neut agent. They do not find a solution in the solution space.}
	\label{fig:solution-mixbed-neut}
\end{figure}

\begin{figure}[h]
	\centering
	\includegraphics[width=0.7\linewidth]{img/scatter_of_proposals}
	\caption{A scatter plot of the proposals of the agents.}
	\label{fig:scatterofproposals}
\end{figure}


	

\subsection{Holonistic agents}

This structure is that of a holon as can be seen in figure~\cref{fig:holonexample}. As shown in the literature it is based on PROSA by \citet{van1998reference}.
\begin{figure}
	\centering
	\includegraphics[width=0.7\linewidth]{img/holon_example}
	\caption{An example of the different negotiation between holons from \citet{beheshti2016negotiations}.}
	\label{fig:holonexample}
\end{figure}

The following facts and rules are part of the Anion.

\begin{enumerate}
	\item
	Knowledge of anion head about the sub-agents:
	\begin{itemize}
		\item {$\{A_1, ..., A_6\}$ can process $a$ amount of water}
		\item {$\{A_1, ..., A_6\}$ needs to be cleaned after $b$ water}
		\item {$\{A_1, ..., A_6\}$ has filtered $c$ amount of water}
		\item {$\{A_1, ..., A_6\}$ needs $d$ base to clean}
		\item {$\{A_1, ..., A_6\}$ needs $e$ time to clean}
	\end{itemize}
	\item
	Currently $x$ amount of water being filtered 
	\item
	Currently $Z \subseteq \{A_1, ..., A_6\}$ filter being used for water filtering
	\item
	Currently $Y \subseteq \{A_1, ..., A_6\}$ filter being used for cleaning
	\item
	Currently $w$ amount of base being used for cleaning
\end{enumerate}

\begin{figure}[h]
	
	\centering
	\begin{tikzpicture}
	
	\node[circle,draw,  minimum size=1cm] (A1) at  (0,0) {A$_1$};
	\node[circle,draw,  minimum size=1cm] (A2) at  (0,-1.5) {A$_2$};
	\node[circle,draw,  minimum size=1cm] (A3) at  (0,-3) {A$_3$};
	\node[circle,draw,  minimum size=1cm] (A4) at  (0,-4.5) {A$_4$};
	\node[circle,draw,  minimum size=1cm] (A5) at  (0,-6) {A$_5$};
	\node[circle,draw,  minimum size=1cm] (A6) at  (0,-7.5) {A$_6$};
	%\draw  (0,-2.5) ellipse (1 and 3.4);
	
	\node[ellipse,  draw, minimum height =9cm, minimum width = 2.5cm ] (A) at (0,-3.75) {Anion};
	
	\end{tikzpicture}
	\caption{Anion head and sub-agents}
	\label{fig:anion-head-sub}
	
\end{figure}


\subsection{Utility function}
\subsubsection{Reservation curve}

\begin{figure}
	\centering
	\begin{tikzpicture}[domain=0.15:4]
	%\draw[very thin,color=gray] (0.01,0.01) grid (3.9,3.9);
	\draw[->] (0.02,0) -- (4.2,0) node[below] {$Water$};
	\draw[->] (0,0.02) -- (0,4.2) node[left] {$Base$};% node[pos=0.25, left] {$200 m^3 / hr$};
	\draw[color=black] plot (\x,{0.07*exp(\x)}) node[left] {$R_A$};
	\end{tikzpicture}
	\label{fig:anionreservationfunction}
	\caption{The reservation function for the Anion filter: the more water is filtered and given, the more base it requires.}
\end{figure}
