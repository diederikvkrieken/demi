%abstract
\begin{abstract}
Businesses around the world change from a centralized hierarchical structure to a decentralized structure. This change, part of Industrie 4.0, requires new methods which ensure that the manufacturing and production processes are still optimized. A possible method is the use of multi-agent systems. Central in the implementation of such a system is the communication, such as negotiation. With different negotiation techniques, processes optimization is achievable. 

In this research the possible negotiation techniques that can be used for the agents to communicate are discussed. Some of these are desirable to optimize processes. A possible solution is the use of multi-issue multi-lateral negotiation, with private utility functions. Using the alternation projection method to negotiate, process optimization should be possible. 

This is tested with a use case and, using the reactive compared to the non-reactive concession strategy, the optimal concession strategy is discussed. It is found that the reactive concession strategy is not as well performing as the non-reactive in respect to the systems optimal (Nash and Pareto) solution, since it can stall while the agreement-zone is non-empty. However, if a single agent uses the reactive strategy, the system performs well.  A possible solution could be the use of different concession strategies, and future research steps could clarify these.

\end{abstract}