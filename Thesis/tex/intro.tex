%Introduction
\chapter{Introduction}
\label{ch:intro}
This thesis is written as part of the Master Artificial Intelligence at the University of Groningen on behalf of the multi-agent systems (MAS) group. The MAS group is part of the Artificial Intelligence and Cognitive Engineering (ALICE) research institute. This group is led by Prof. dr. L.C. (Rineke) Verbrugge.
   
\section{Introduction in Production, AI and the Use Case Environment}

Currently a lot of research is conducted in Artificial Intelligence (AI) and how to apply this in business context. One field of interest, which is researched in this thesis, is the usage of a multi-agents system in production and manufacturing using negotiation.

\subsection{Production and Manufacturing}
Production is the process of converting inputs into outputs. It is one of the economic pillars on which the economic markets are driven. By creating extra value from basic commodities, a (perceived) contribution to the well-being of individuals is conceivable. Manufacturing is a specific subsidiary of production, and is the process of converting (raw) material into semi and/or (finished) end products by making use of various processes, machines and energy. Thus, every type of manufacturing can be production, but not every type of production is manufacturing.

The production and manufacturing industry is and will be one of the wealth generators of the world economy \citep{monostori2006agent}, and is characterised by the production of commodities that have value and contribute to the well-being of individuals.
% produce commodities which have value and contribute to well-being of individuals It is one of the 

In the industrial production world, a fourth industrial revolution is going on, which enables the world to think about new production processes. The first industrial revolution was the use of steam power to mechanize production. In the second industrial revolution, the use of electric power allowed for assembly lines, resulting in mass production. The third revolution used electronics and information technology to automate production. Now a fourth industrial revolution, also called Industry 4.0\footnote{This revolution has multiple terms in multiple countries. For example, `Industrie 4.0' in Germany, `Smart Manufacturing' or `Smart Industry' in the Netherlands, or the `Industrial Internet Consortium' in the U.S.A. In this thesis the term `Industry 4.0' will be used.}, is building on the third, and is called the digital revolution. It is characterized by a fusion of technologies that is blurring the lines between the physical and digital worlds, and the convergence of IT and OT \citep{leitao2016smart}.

Throughout this thesis, the terms production and manufacturing will be used interchangeably. This does not mean that the terms are interchangeable in general, since in the industry there is a difference. However, for this research, due to the similarity in the sense of the processes, no separation is required. This is supported by the exchangeability of the terms in the literature.

\subsection{Artificial Intelligence}
The research is based on an intelligent multi-agent system (MAS) which consists of agents which act and react on their environment in both a physical and an IT way. For the intelligent agents it is possible, by understanding the system and by negotiating, to come up with a (near-) optimal production plan. Furthermore it can optimally allocate resources taking in consideration possible maintenance and downtime, based on real-time data acquisition, analysis, negotiations and decentralized autonomous decision making. Such intelligence is an example of a typical MAS where artificial intelligence may include methodical, functional, and procedural approaches, algorithmic search and/or reinforcement learning.

\subsection{Ecosystem of the Case Study}
\label{sec:intro_ecosystem}
In this thesis a new model is constructed based on negotiation in an intelligent multi-agent system. An application of this new model is tested and modelled based on a plant that creates de-mineralized water. By removing all the ions from common water, de-mineralized water is obtained. This water is used for multiple processes and applications. In this plant specifically it is used for the steam turbines, which generate electricity. By burning the by-product, heat is generated, which creates steam to power the turbines.

Minerals are removed from water in multiple production steps. Most common, and as is implemented in the plant described, is to first remove the positively charged minerals in so called anions. After this, the negatively charged ions are removed with a cation filter. To ensure that all ions are removed, a final combined ``Mixbed'' is used. Here a combination of an anion and cation filter removes the residues.

These filters have to be cleaned every few hours to ensure that proper demineralization occurs. By optimizing the production planning and/or resource allocation, real-time usage of the cleaning resources is possible, including the optimal location, resulting in minimal waste.

\section{Thesis outline}
In \Cref{chp:problem} an overview of the problem is given. \Cref{ch:literature} will explain the literature regarding manufacturing and negotiation. This also includes an overview of the methods. Based on these methods a framework is introduced, after which a knowledge gap is determined. This knowledge gap, focussing on negotiation, is used to design and implement our model, the foundation of \Cref{ch:design}. In \Cref{ch:eval} the model is tested and evaluated. From this it is possible to conclude and detect further use as described in \Cref{ch:conc}.
\begin{enumerate}
	
	\item
	Introduction 
	\item
	Problem Definition and Research Goal
	\item
	Literature Study and Theoretical Framework
	\item
	Research Design and Application
	\item
	Simulation comparison and Evaluation
	\item
	Conclusion and Further work
\end{enumerate}

%Outline scriptie
%1 Introduction 1
%1.1 Introduction into manufacturing, AI and ecosystem
%1.1.1 Manufacturing
%1.1.2 Artiffcial Intelligence 
%1.1.3 Ecosystem 
%1.2 Thesis outline 
%2 Problem Definition & Research Goal 
%2.1 Problem analysis 
%2.2 Area of Application
%2.3 Relevance 
%2.3.1 Scientifc relevance
%2.3.2 Business relevance
%2.4 Research Goal.
%2.5 Research Approach.
%2.6 Research process
%2.6.1 Evaluation method
%2.7 Research questions
%
%
%3 Literature Study & Theoretical framework 
%3.1 Manufacturing process
%3.1.1 Asset management
%3.1.2 e-manufacturing 
%3.1.3 Predictive maintenance systems
%3.1.4 Process control
%3.1.5 Programmable Logic Controller
%3.1.6 Real time Monitor
%3.2 Agent Solutions
%3.2.1 Object oriented programming 
%3.2.2 multi-agent systems
%3.2.3 Holonic Systems 
%3.2.4 Scheduling/Planning 
%3.2.5 Negotiation
%3.2.6 Negotiation in Manufacturing 
%3.2.7 Internet of Things
%3.3 Framework
%3.3.1 Size
%3.3.2 Modularity
%3.3.3 Time Scale
%3.3.4 Solution Quality 
%3.3.5 Complexity
%4 Research Design & Implementation 
%4.1 Mapping of Literature on use case
%4.2 Mapping of Negotiation
%4.3 Evaluation.
%5 Simulation comparison & evaluation
%5.1 Real world use.
%5.1.1 Technical requirements
%5.1.2 Roadmap of Decentralized control system (DCS) 2.0 
%6. Conclusion and Discussion
%
%